\documentclass[10pt]{article}
\usepackage{amsmath}
\usepackage{setspace}
\usepackage{hyperref}
\usepackage{booktabs}

\setlength{\textheight}{9in} \setlength{\topmargin}{-.5in}
\setlength{\textwidth}{6.5in} \setlength{\oddsidemargin}{0in}
\setlength{\evensidemargin}{0in}

\title{Syllabus \\ COMP 151 \\ Introduction to Programming}
\author{  }
\date{Fall 2024}

\begin{document}
\maketitle

\section{Logistics}
\begin{itemize}
\item \textbf{Where: } Center for Science and Business 309.
\item \textbf{When: }
\begin{itemize}
  \item Class: MWF 8--8:50am
  \item Lab: M 2--3:50pm
\end{itemize}
\item \textbf{Instructor: } Logan Mayfield
\begin{itemize}
\item \textit{Office: } Center for Science and Business (CSB), Room 344
\item \textit{Phone: } 309-457-2200 % chktex 8
\item \textit{Website: } \url{http://jlmayfield.github.io/}
\item \textit{Email: } lmayfield \textit{at} monmouthcollege \textit{dot} edu
\item \textit{Office Hours: }  M 9-10am. Tu, 10:30am - 11:30am. W, 2-3pm. Th, 10:30-11:30am. F, 2-3pm. By appointment.
\end{itemize}
\item \textbf{Website: } \url{http://jlmayfield.github.io/teaching/COMP151/}
\item \textbf{Credits: } 1 course credit
\end{itemize}
\emph{Note: This Syllabus is subject to change based on specific class needs. Significant deviations from the syllabus will be discussed in class.}


\section{Description, Content, and Learning Goals}

Introduction to Programming teaches basic programming skills that are applicable to a variety of disciplines and also acts as a bridge to continued studies in Computer Science. Students will work with the Python programming language in order to solve basic and moderately complex problems using computing. By the end of the course students will be able to read and develop computer programs utilizing the following programming concepts: basic data types and encoding, variables and scope, array and list data structures, if statements and conditional execution, loops and iteration, functions, and object types.

\subsection{Textbook}

\noindent
Hrehirchuk, Matthew, et. al.. \textit{Foundations of Python Programming: Functions First}.  runestoneinteractive.org. \url{https://runestone.academy/ns/books/published/foppff/fopp-ff-3.html} % chktex 8


\section{Workload}
% number of/details on midterms, finals, project, homeworks, quizes, etc


Time spent on work for this course will likely vary by student and will, in general, vary week to week. On average, this course should require about 13 hours of work per week per student.  The following table provides a rough estimate of the distribution of this time over different course components.
\begin{center}
\begin{tabular}{ll}
\underline{Assignment Type} & \underline{Time/week} \\
Class Time \& Labs       & 5 hours/week \\
Self-Study \& Reading & 2-3 hours/week \\
Game Dev Project           & 3 hours/week \\
Homework Problems   & 1-2 hours/week \\
\bottomrule
 & 11-13 hours/week
\end{tabular}
\end{center}


The course uses fairly standard types of assignments: labs, homework problems, exams, projects, etc. The number of such assignments you can expect to complete is given below along with individual assignment type descriptions.
\begin{center}
  \begin{tabular}{ll}
    \underline{Category} & \underline{Number of Assignments} \\
    Exams & 5--7 \\
    Game-Dev Project & 1 \\
    Labs & 8 \\
    Homework & 8 \\
    Portfolio Review \& Self Evaluation Meetings & 4-5
  \end{tabular}
\end{center}

\subsection*{Labs}

Labs are hands on, active learning periods. You'll dive into new ideas and new techniques through programming problems done with a partner and with the instructor there to help. Where other assignments are more about reinforcement and assessment of things previously learned, lab time is a time to dive into new things. \textit{They goal of a lab is to explore, to play, to break things then fix them, to ask questions, and to otherwise uncover everything you can about a new corner of programming.}

Expect to do labs on a weekly basis. They will typically take not more than the two hour lab period and will never require that you work on them outside of that lab period. We'll typically go over lab problems in detail during the class period immediately following the lab.

\subsection*{Homework}

\textit{Homework is practice. It's drills. It's activities designed largely to reinforce and strengthen your understanding of material learned through reading, through class time, and through lab.} Homework lets you gut check what you think you know and how well you know it. They afford to you the time to do the work and then check that work using other references, including classmates. They are meant to be low stakes assignments.  Just don't confuse low stakes with low importance. Practice, repetition, and reinforcement through homework is important for learning. It is what gets you ready for exams, projects, and programming outside of this class.

Homework assignments will typically involve completing problems in the online text as well as other problems related to the current material. The class will work-on and discuss problems of interest between the assignment and due date of the problem set. A review of homework problems will typically proceed exams.

\subsection*{Exams}

Exams are meant to test your understanding of and ability to apply ideas covered previously in the course. They are a gut check the current state of your learning. They let you answer the question, ``Do I know the material as well as I think I know the material?''  Compared to homework, they are higher stakes assessments of your learning because you typically lack the safety net that is your notes, the text, and other references.

For the most part, exams will be done in class or in lab and will be announced ahead of time in order for you to prepare.  Expect these exams to take all or most of a class period and involve multiple questions or a multi-part problem. On a few occasions we'll have small pop-exams that are unannounced, involve one or two quick questions, and will only take up a small portion of the start of a class period.

\iffalse
\subsection*{Projects}

\textit{You should look at the projects like game day or the big performance. They are, in large part, what we're preparing for with other smaller assignments.} You should give them your best effort and a great deal of your time. You'll learn and grow the most as a programmer by really digging in and engaging in the projects and all of the challenges you'll face when working on them.

Projects are large scale programming assignments done over the course of two weeks. They will draw on everything you've done and learned in the class up to that point. They will also typically involve some new ideas that you must navigate and integrate into your work. Some lab and class time will be used to work on the projects, but you should plan for the bulk of your work on them to take place outside of class and lab.
\fi

\subsection*{Game-Dev Project}

Much of our work is transient. You write a bit of code for a homework or lab then walk away from it. In practice, programmers live with code for long periods of time. They update it, fix previously undiscovered bugs, and generally maintain that code over months and years.  To replicate this, and to give you a chance to be creative with your programming, we'll be incrementally developing a game over the course of the course of the semester.  As we expand our programmer's toolbox, we'll expand the scope and features of our games.  What's more, we'll be sharing our games with one another and possibly with the campus at large. \textit{You should treat this project as \textbf{the assignment} for the course because the process of developing it will give you the chance to cover all the competency goals of the course and showcase everything you've learned.}


\subsection*{Portfolio Review \& Self-Evaluation}

Self-reflection and self-evaluation is a critical component of learning and vital to a growth mindset.
We will keep a portfolio of the work you do throughout the semester. Much of this will be done automatically
by our assignment management and version control software. At regular intervals throughout the semester you will meet, one-on-one, with me to \textit{present your porfolio}, review items from your portfolio that best 
gauge how well you're doing at meeting the course goals and expectations, and discuss how that success maps to 
a letter grade. For more details about the process visit \url{https://jlmayfield.github.io/teaching/ungrading/howto-portfolio}.


\section{Ungrading \& Final Grades}

This class is largely ungraded. That means your assignments will not be graded for points and your final grade
is not determined by a point-based, numerical grading system. You will get feedback on your work but you will
see points on nothing. You don't earn points for doing work or getting something correct nor do you lose points
for getting something wrong. We're here to learn. Doing the work is how we do that and getting things wrong
some or most of the time is part of learning. For more details visit \url{https://jlmayfield.github.io/teaching/ungrading/howto-portfolio}.

\subsection{Self-Evaluation \& Final Course Grades}

Throughout the semester you'll be asked to engage in regular self-evaluation. This process is described in
detail in additional documentation. Part of the process includes you self-assigning a course grade based on
your self-evaluation. Your self-evaluation and self-assigned grade are then discussed with me in a one-on-one
meeting during which we'll agree upon your current grade. The key here is that \textit{your self-evaluation
and self-assigned grade begins the conversation, not my assigned points.}

Below are some general rules of thumb we'll try to stick to when talking about grades. They relate grades to
course competency expectations and Monmouth College policy.
\begin{itemize}
  \item \textbf{A} - Exceeding course expectations.
  \item \textbf{B} - Meeting and occasionally exceeding course expectations.
  \item \textbf{C} - Meeting course expectations. \textit{This is the minimum grade required to continue on to COMP152. So, a C means you can be successful in a class that builds upon the things learned in this class.}
  \item \textbf{C-} - Mostly meeting course expectations. \textit{This is the minium grade that counts towards a major.}
  \item \textbf{D} - Occasionally meeting course expectations, but mostly not. \textit{Grades in the D range earn credit towards graduation but fall below GPA requirements.}
  \item \textbf{F} - Did not meet course expectations.
\end{itemize}

My hope is that the self-evaluation and self-directed grading process provides a lot of flexibility in terms
of how you can achieve success in this course and meet your grade goals. If you ever have questions or concerns
about self-evaluations and grades, then I'm more more than willing to discuss them with you at any time.

\subsubsection{Participation, Attendance, \& Timely Work}

I do not have strict attendance and deadline policies, per se, but I do have clear expectations. These
expectations are baked into the dispositional attribute of the course competencies. This attribute
includes things like being \textit{professional, responsible, responsive, and self-directed.}

As far as I'm concerned, signing up for this class means you agree to coming to class and lab,
being on time for class and lab, doing assigned work and submitting it on time, and generally participating
in all the class has to offer.  That being said, life happens and people have different priorities.
You might need to miss class or extend a deadline.  So long as you communicate with me about it, as a
professional would with a co-worker, then we won't have a problem. If you simply skip class without
warning, always show up late, or regularly fail to do assigned work in a timely manner, then I expect that
those failures to meet dispositional expectations to be reflected in your self-evaluation.

There is one exception to my ``no grade-based policy'' on assignments and deadlines and that is the
self-evaluations and reflections. The self-evaluation process is critical to this class and in no way
optional. \textbf{If you fail attend the portfolio review meetings or always show up completely un-prepared
then I reserve to give you a final grade of D or lower for the course.} You'll find I can be pretty relaxed
about a lot of other assignments and deadlines, but I draw the line at the self-evaluation process.


\subsubsection*{Academic Honesty}

You don't learn by trying to pass off someone's work as your own. In an ungraded class it makes even less sense to cheat and steal work from somewhere else.  There are no points, you gain nothing from it and you certainly will learn nothing from it. In this ungraded class, academic dishonesty is still not tolerated.

From the Monmouth College Academic Honesty Policy:
\begin{quote}
  ``We view academic dishonesty as a threat to the integrity and intellectual mission of our institution. Any breach of the academic honesty policy - either intentionally or unintentionally - will be taken seriously and may result not only in failure in the course, but in suspension or expulsion from the college. It is each student’s responsibility to read, understand and comply with the general academic honesty policy at Monmouth College, as defined here in the Scots Guide, and to the specific guidelines for each course, as elaborated on the professor’s syllabus.''

  ``The following areas are examples of violations of the academic honesty policy:
  \begin{enumerate}
  \item Cheating on tests, labs, etc;
  \item Plagiarism, i.e., using the words, ideas, writing, or work of another without giving appropriate credit;
  \item Improper collaboration between students, i.e., not doing one’s own work on outside assignments specified as group projects by the instructor;
  \item Submitting work previously submitted in another course, without previous authorization by the instructor.''
  \end{enumerate}

  ``Please note that this list is not intended to be exhaustive.''
\end{quote}

In this course, any violation of the academic honesty policy will have varying consequences depending on the severity of the infraction as judged by the instructor.  Expect violations to be reported to the appropriate Dean and to weaken your case for higher grades at the end of the course. Severe violations can result in an F for the course and expulsion from the course. Do your own work. If you even think something you're doing could be construed as academically dishonest, then ask for guidance and clarification first.


\subsection*{Generative AI Policy}

In general, you are not allowed to use generative AI to generate any portion of your work.  No ChatGPT, no Github CoPilot, none of that.  If you're turning the work into me with your name on it, then it should be your work. Writing or tweaking a prompt to generate a solution is not the same as actually generating the solution on your own. We may explore the use of these tools in developing software systems and writing code, but in such cases you will be explicitly told that you can use AI. If you use AI to generate work without the express permission of the instructor, then you will have committed an act of academic dishonesty and will face appropriate consequences (see above).

While you cannot use AI to generate work, you can use AI as a study guide and a tool to assist in learning. This includes getting new study problems from an AI, having AI summarize or paraphrase portions of a reading when studying (not as part of an assignment!), and otherwise finding creative uses for AI.  If, at any point, you are in doubt about whether or not your use of AI is appropriate for the course, ask.  In fact, I'd love to hear about creative ways you're using AI.  

\section{Academic Support \& Accessibility}

\subsection*{Support Services}
The Academic Support and Accessibility Services Office offers free resources to assist Monmouth College students with their academic success. Programs include Supplemental Instruction for difficult classes, Drop-In and appointment tutoring, and individual Academic Coaching. Our office is here to help all students excel academically, since every student can work toward better grades, practice stronger study skills, and manage their time better. Please email academicsupport@monmouthcollege.edu for assistance.

\subsection*{Accessibility Services}
If you have a disability and/or medical/mental health condition or had academic accommodations in high school or another college, you may be eligible for academic accommodations at Monmouth College under the Americans with Disabilities Act (ADA). Monmouth College is committed to equal educational access. To discuss any of the services offered, please call or meet with Jennifer Sanberg, Associate Director of Academic Support and Accessibility Services. The ASAS office is located on the first floor of the Hewes Library, opposite Einstein’s Bros Bagel. They can be reached at 309-457-2257 or via email at: academicsupport@monmouthcollege.edu

\subsection{Calendar}

\textit{This calendar aspirational and is subject to change based on the circumstances of the course. A more detailed, regularly updated calendar can be found on the course website. }

\begin{center}
\begin{tabular}{lllll}
\underline{Week} & \underline{Dates} & \underline{Assignments Due} & \underline{Chapter(s)} & \underline{Notes} \\
1 & 8/21--8/23  & Hwk 1. & Ch 1. &  \\
2 & 8/26--8/30 &  Hwk 2. Lab 1. & Ch 2--3 & \\
3 & 9/2--9/6 & (Take Home) Lab 2. Exam 1. & Ch 3 & NO CLASS M. - Labor Day. \\
4 & 9/9--9/13  & Lab 3. Hwk 3.  &  Ch 4. &  \\
5 & 9/16--9/20 & Lab 4. Exam 2. & Ch 4. Ch 6.1--6.3.  & \\
6 & 9/23--9/27 & Game 1 Lab. & Ch 5. PyGame. & \\
7 & 9/30--10/4 & Game 1 Due. Hwk 4. & Ch 5. Ch 6.4+  & \\
8 & 10/7--10/9 & Game 2 Lab. Exam 3. & Ch 6.4+ & FALL BREAK Th-F  \\
9 & 10/14--10/18 & Game 2 Lab. & Ch 7 & \\
10 & 10/21--10/25  & Lab 5. Game 2 Due. Hwk 5. & Ch 7--8 & \\
11 & 10/28--11/1 & Lab 6. Hwk 6. & Ch 8--9. & \\
12 & 11/4--11/8 & Lab 7. Exam 4. & Ch 9--10  & \\
13 & 11/11--11/15 & Lab 8. Exam 6. &  Ch 10.  & \\
14 & 11/18--11/22 & Game 3 Lab. &  & \\
15 & 11/25--11/26 & Game 3 Lab. & Ch 11-13 & THANKSGIVING W-F \\
16 & 12/2--12/6 & & Game 3 Due. & NO CLASS F. - Reading Day Th.\\
17 & 12/10 & Exam 7. Tues. 12/10 6:30 -- 9:30 pm & &  \\
\end{tabular}
\end{center}

\end{document}
