\documentclass[10pt]{article}
\usepackage{amsmath}
\usepackage{setspace}
\usepackage{hyperref}
\usepackage{booktabs}

\setlength{\textheight}{9in} \setlength{\topmargin}{-.5in}
\setlength{\textwidth}{6.5in} \setlength{\oddsidemargin}{0in}
\setlength{\evensidemargin}{0in}

\title{Syllabus \\ COMP 151 \\ Introduction to Programming}
\author{  }
\date{Fall 2022}

\begin{document}
\maketitle

\section{Logistics}
\begin{itemize}
\item \textbf{Where: } Center for Science and Business 309.
\item \textbf{When: }
\begin{itemize}
  \item Class: MWF 8--8:50am
  \item Lab: M 2--3:50pm
\end{itemize}
\item \textbf{Instructor: } Logan Mayfield
\begin{itemize}
\item \textit{Office: } Center for Science and Business (CSB), Room 344
\item \textit{Phone: } 309-457-2200 % chktex 8
\item \textit{Website: } \url{http://jlmayfield.github.io/}
\item \textit{Email: } lmayfield \textit{at} monmouthcollege \textit{dot} edu
\item \textit{Office Hours: }  By appointment.
\end{itemize}
\item \textbf{Website: } \url{http://jlmayfield.github.io/teaching/COMP151/}
\item \textbf{Credits: } 1 course credit
\end{itemize}
\emph{Note: This Syllabus is subject to change based on specific class needs. Significant deviations from the syllabus will be discussed in class.}


\section{Description, Content, and Learning Goals}

Introduction to Programming teaches basic programming skills that are applicable to a variety of disciplines and also acts as a bridge to continued studies in Computer Science. Students will work with the Python programming language in order to solve basic and moderately complex problems using computing. By the end of the course students will be able to read and develop computer programs utilizing the following programming concepts: basic data types and encoding, variables and scope, array and list data structures, if statements and conditional execution, loops and iteration, functions, and object types.

\subsection{Textbook}

\noindent
Elkner, Jeffery, et. al.. \textit{Foundations of Python Programming}.  runestoneinteractive.org. \url{https://runestone.academy/ns/books/published/fopp/index.html} % chktex 8


\section{Workload}
% number of/details on midterms, finals, project, homeworks, quizes, etc


Time spent on work for this course will likely vary by student and will, in general, vary week to week. On average, this course should require about 13 hours of work per week per student.  The following table provides a rough estimate of the distribution of this time over different course components.
\begin{center}
\begin{tabular}{ll}
\underline{Assignment Type} & \underline{Time/week} \\
Class Time \& Labs       & 5 hours/week \\
Self-Study \& Reading & 2-3 hours/week \\
Projects             & 3 hours/week \\
Homework Problems   & 1-2 hours/week \\
\bottomrule
 & 11-13 hours/week
\end{tabular}
\end{center}


This uses fairly standard types of assignments: labs, homework problems, exams, projects, etc. The number of such assignments you can expect to complete is given below along with individual assignment type descriptions.
\begin{center}
  \begin{tabular}{ll}
    \underline{Category} & \underline{Number of Assignments} \\
    Exams & 5--7 \\
    Projects & 2 \\
    Labs & 10 \\
    Homework & 8 \\
    Self-Evaluation Letters & 4 \\
  \end{tabular}
\end{center}

\subsection*{Labs}

Labs are hands on, active learning periods. You'll dive into new ideas and new techniques through programming problems done with a partner and with the instructor there to help. It's a time to learn new things where other assignments are more about reinforcement and assessment of things previously learned. \textit{They goal of a lab is to explore, to play, to break things then fix them, to ask questions, and to otherwise uncover everything you can about a new corner of programming.}

Expect to do labs on a weekly basis. They will typically take not more than the two hour lab period and will never require that you work on them outside of that lab period. We'll typically go over lab problems in detail during the class period immediately following the lab.

\subsection*{Homework}

\textit{Homework is practice. It's drills. It's activities designed largely to reinforce and strengthen your understanding of material learned through reading, through class time, and through lab.} Homework lets you gut check what you think you know and how well you know it. They afford to you the time to do the work and then check that work using other references, including classmates. They are meant to be low stakes assignments.  Just don't confuse low stakes with low importance. Practice, repetition, and reinforcement through homework is important for learning. It is what gets you ready for exams, projects, and programming outside of this class.

Homework assignments will typically involve completing problems in the text as well as other problems related to the current material. The class will work-on and discuss problems of interest between the assignment and due date of the problem set. A review of homework problems will typically proceed exams.

\subsection*{Exams}

Exams are meant to test your understanding of and ability to apply ideas covered previously in the course. They are a gut check the current state of your learning. They let you answer the question, ``Do I know the material as well as I think I know the material.''  Compared to homework, they are higher stakes assessments of your learning because you lack the safety net that is your notes, the text, and other references.

For the most part, exams will be done in class or in lab and will be announced ahead of time in order for you to prepare.  Expect these exams to take all or most of a class period and involve multiple questions or a multi-part problem. On a few occasions we'll have small pop-exams that are unannounced, involve one or two quick questions, and will only take up a small portion of the start of a class period.

\subsection*{Projects}

\textit{You should look at the projects like game day or the big performance. They are, in large part, what we're preparing for with other smaller assignments.} You should give them your best effort and a great deal of your time. You'll learn and grow the most as a programmer by really digging in and engaging in the projects and all of the challenges you'll face when working on them.

Projects are large scale programming assignments done over the course of two weeks. They will draw on everything you've done and learned in the class up to that point. They will also typically involve some new ideas that you must navigate and integrate into your work. Some lab and class time will be used to work on the projects, but you should plan for the bulk of your work on them to take place outside of class and lab.

\subsection*{Self-Evaluation Letters}

Self-reflection and self-evaluation is a critical component of learning and vital to a growth mindset. You'll write four self-evaluation letters to me through out the course of the semester. In these letters you'll evaluate the state of your learning, present evidence of successes, examples of ongoing challenges, and address how well you believe you're meeting course and personal goals and expectations. As you'll read below, these letters and the conversations we have as a result of the letters will, by and large, determine your grade in the course.

Letters will be one to two pages in length and will be submitted at the start of class, after the first project, at midterm, and at the end of the semester along with the final project.


\section{Ungrading \& Final Grades}

This class is ungraded. That means your assignments will not be graded and your final grade is not determined by a point-based, numerical grading system. You will get feedback on your work but you will see points on nothing. You don't earn points for doing work or getting something correct nor do you lose points for getting something wrong. We're here to learn. Doing the work is how we do that and getting things wrong some or most of the time is part of learning. If I could do away with assigning a final grade I would, but the college requires it.

\subsection{Self-Evaluation \& Final Course Grades}

Throughout the semester you'll be asked to engage in regular self-evaluation, the purpose of which is to evaluate your learning and begin a dialog about your progress with the instructor. The result of the evaluation will be a letter to the professor. This process is also where and how your final grade will be determined. As a part of your self-evaluation you'll give yourself a grade and discuss why you think that grade is appropriate given what you presented in your self-evaluation of learning. You start the grading conversation. After having read your self-evaluation, I will discuss with you the proposed grade and we'll come to an agreement on your current grade.

As a class and in our post-self-evaluation dialogues, we'll be discussing some standards and expectations for determining final grades in the absence of a point-based system. The cornerstones of these expectations are the things we need to do in order to learn and grow: meaningful, regular, and intensional work in order to meet established learning goals and standards.


\subsubsection{Doing Good Work: Participation, Attendance, \& Timely Work}

You are expected to be in class and in lab. Missing class and lab is like skipping practice for a sport, for a play, for a musical performance, or for anything else in which people expect to prepare ahead of time for an event. Learning in this class is meant to be a social exercise. We will explore new ideas as a class, ask and answer questions of one another, and generally use each member of the class as a resource. If people don't show up, then it not only robs them of the chance to grow through the day's activities but it also robs the members of the class of the chance to benefit from their unique perspective as an individual. That being said, missed classes are often unavoidable. The key is missing for a good, largely unavoidable reasons and trying to keep the absences to a minimum.

Assignments will be given, collected, and returned when it best benefits your learning. By doing an assignment late you risk getting out of sync with the class and not getting as much out of the assignment as you otherwise could. It can also make you an unequal contributor to group work as you will not have practiced and reinforced the same material as your classmates. Still, things happen and we can't always meet deadlines.  What's important is that you make all reasonable efforts to get all assignments done and in on time.


\subsubsection*{Academic Honesty}

You don't learn by trying to pass off someone's work as your own. In an ungraded class it makes even less sense to cheat and steal work from somewhere else.  There are no points, you gain nothing from it and you certainly will learn nothing from it. In this ungraded class, academic dishonesty is still not tolerated.

From the Monmouth College Academic Honesty Policy:
\begin{quote}
  ``We view academic dishonesty as a threat to the integrity and intellectual mission of our institution. Any breach of the academic honesty policy - either intentionally or unintentionally - will be taken seriously and may result not only in failure in the course, but in suspension or expulsion from the college. It is each student’s responsibility to read, understand and comply with the general academic honesty policy at Monmouth College, as defined here in the Scots Guide, and to the specific guidelines for each course, as elaborated on the professor’s syllabus.''

  ``The following areas are examples of violations of the academic honesty policy:
  \begin{enumerate}
  \item Cheating on tests, labs, etc;
  \item Plagiarism, i.e., using the words, ideas, writing, or work of another without giving appropriate credit;
  \item Improper collaboration between students, i.e., not doing one’s own work on outside assignments specified as group projects by the instructor;
  \item Submitting work previously submitted in another course, without previous authorization by the instructor.''
  \end{enumerate}

  ``Please note that this list is not intended to be exhaustive.''
\end{quote}

The complete Monmouth College Academic Honesty Policy can be found on the College web page by clicking on ``Student Life'' then on ``Scot’s Guide'' in the navigation bar to the left, then ``Academic Regulations'' in the navigation bar at the left.  Or you can visit the web page directly by typing in this URL: \url{https://ou.monmouthcollege.edu/life/residence-life/scots-guide/academic-regulations.aspx}

In this course, any violation of the academic honesty policy will have varying consequences depending on the severity of the infraction as judged by the instructor.  Expect violations to be reported to the appropriate Dean and to weaken your case for higher grades at the end of the course. Severe violations can result in an F for the course and expulsion from the course. Do your own work. If you even think something you're doing could be construed as academically dishonest, then ask for guidance and clarification first. 

\subsubsection*{Evidence of Learning: Course Competencies}

This course has a set of competencies that combine areas of knowledge, levels of skill, professional dispositions with a specific task. The result is a broad set of outcomes for the class and the benchmarks by which you can and will assess your learning and overall success in this course. As you progress through the course, you should be able to point to specific assignments, parts of assignments, or things you did in service of completing an assignment as evidence that you are meeting or exceeding the course competencies.


\section{Academic Support \& Accessibility}

\subsection*{Support Services}
The Academic Support and Accessibility Services Office offers free resources to assist Monmouth College students with their academic success. Programs include Supplemental Instruction for difficult classes, Drop-In and appointment tutoring, and individual Academic Coaching. Our office is here to help all students excel academically, since every student can work toward better grades, practice stronger study skills, and manage their time better. Please email academicsupport@monmouthcollege.edu for assistance.

\subsection*{Accessibility Services}
If you have a disability and/or medical/mental health condition or had academic accommodations in high school or another college, you may be eligible for academic accommodations at Monmouth College under the Americans with Disabilities Act (ADA). Monmouth College is committed to equal educational access. To discuss any of the services offered, please call or meet with Jennifer Sanberg, Associate Director of Academic Support and Accessibility Services. The ASAS office is located on the first floor of the Hewes Library, opposite Einstein’s Bros Bagel. They can be reached at 309-457-2257 or via email at: academicsupport@monmouthcollege.edu

\subsection{Calendar}

\textit{This calendar aspirational and is subject to change based on the circumstances of the course. A more detailed, regularly updated calendar can be found on the course website. }

\begin{center}
\begin{tabular}{llll}
\underline{Week} & \underline{Dates} & \underline{Assignments Due} & \underline{Chapter(s)}\\
1 & 8/24 --- 8/26  &  &  1\\
2 & 8/29 --- 9/2 & Lab 1. Hwk 1. Eval Letter 1. & 2,3.1--3.8\\
3 & 9/5 --- 9/9 &  Hwk 2. &  4.1--4.4, 5  \\
4 & 9/12 --- 9/16  & Lab 2. Hwk 3. Exam 1.  &   4,5 \\
5 & 9/19 --- 9/23 & Lab 3.  & 6 \\
6 & 9/26 --- 9/30 & Lab 4. Hwk 4. Eval Letter 2. & 7.1--7.7, 8.1, 12.1--12.5 \\
7 & 10/3 --- 10/7 & Lab 5. Hwk 5. &   8 \\
8 & 10/10 --- 10/14 & Project 1. Exam 2. FALL BREAK (Th-F). &   \\
9 & 10/17 --- 10/21  & Lab 6. & 12.6--12.15,9.1--9.7  \\
10 & 10/24 --- 10/28  & Lab 7. Hwk 6. &  9.8--9.14,7.8--7.12  \\
11 & 10/31 --- 11/4 &  Lab 8. Hwk 7. Exam 3. &  14 \\
12 & 11/7 --- 11/11 &  Lab 9. Hwk 7. Eval. Letter 3. & 10 \\
13 & 11/14 --- 11/18 & Lab 10. Hwk 8.  &  11 \\
14 & 11/21 --- 11/25 &  Exam 4. (THANKSGIVING W-F) Project 2. &  \\
15 & 11/28 --- 12/2 &   &  18,21,17 \\
16 & 12/5 --- 12/10 & Project 2(READING DAY. Th) Exam 5 (Friday 12/9). &  \\
17 & 12/12 --- 12/14 & Eval Letter 4.
\end{tabular}
\end{center}

\end{document}
