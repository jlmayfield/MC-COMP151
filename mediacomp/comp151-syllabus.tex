\documentclass[10pt]{article}
\usepackage{amsmath}
\usepackage{setspace}
\usepackage{hyperref}
\usepackage{booktabs}

\setlength{\textheight}{9in} \setlength{\topmargin}{-.5in}
\setlength{\textwidth}{6.5in} \setlength{\oddsidemargin}{0in}
\setlength{\evensidemargin}{0in}

\title{Syllabus \\ COMP 151 \\ Introduction to Programming \\ \textit{Remote Learning Edition}}
\author{  }
\date{Spring 2021}

\begin{document}
\maketitle

\section{Remote Learning Expectations}

Due to the ongoing COVID-19 pandemic, this course will be delivered remotely. The plan
is to retain most of the regular meetings and opporotunities for help and direct instrution
but to carry out these activities via online tools. Here's what that means, what you can expect,
and what will be expected of you for this mode of delivery.

\begin{itemize}
  \item Remote contact information for the instructor and Zoom meeting links will be posted to the
  course LMS.

  \item Assignments and their subsequent grades will always be posted to on the main course LMS. Submissions will often be managed through another online platform, but all roads begin and end at the LMS.

  \item Class and Lab sessions will meet at the usual time but will be carried out
  via Zoom. \textit{Attendance expectations for these sessions are the same as
  if the class were meeting in person.}

  \item It is not a requirement of the course that you always have your camera on nor
  that you speak up on your microphone. I respect your privacy and understand that its not
  always easy or possible to be online in that way. That being said, \textit{the expectation
  is that ``cameras on'' is the norm}. If you need a camera break or cannot run your camera for
  whatever reason, then \textit{you should set as your Zoom profile image a photo of you or,
  alternatively, an appropriate for all ages image that represents you}. No non-discript black
  boxes in this class. Be present and make your presence known every class period.

  \item Expect many class sessions to be recorded for the benefit of students that
  are unable to attend class due to an excused absense or illness. If you do not wish
  to be recorded, please contact the instructor ASAP to work something out.

  \item Office hours will be carried out via Zoom. Whenever possible, give the
  instructor a heads up if you need or want to meet. If you log into the Zoom meeting
  and the session isn't running or nobody is there, then email or txt and the instructor
  will hop right on.

\end{itemize}

If you have any concerns about getting access to the technology you need to keep up with
remote meetings or generally have questions about class exceptions for this semseter,
then please do not hesitate to ask. The goal here is flexiblity and we try to make
things work for whatever you situation happens to be.

\newpage

\section{Logistics}
\begin{itemize}
\item \textbf{Where: } The Cloud.
\item \textbf{When: }
\begin{itemize}
  \item Class: MWF 8--8:50am
  \item Lab: T 2--4:50pm
\end{itemize}
\item \textbf{Instructor: } Logan Mayfield
\begin{itemize}
\item \textit{Office: } Center for Science and Business (CSB), Room 344\footnotemark[2]
\item \textit{Phone: } 309-457-2200 % chktex 8
\item \textit{Website: } \url{http://jlmayfield.github.io/}
\item \textit{Email: } lmayfield \textit{at} monmouthcollege \textit{dot} edu
\item \textit{Office Hours: }  By appointment.
\end{itemize}
%\item \textbf{Website: } \url{http://jlmayfield.github.io/teaching/COMP151/}
\item \textbf{Credits: } 1 course credit
\end{itemize}
\emph{Note: This Syllabus is subject to change based on specific class needs. Significant deviations from the syllabus will be discussed in class.}

\footnotetext[2]{Do not expect this office to be occupied unless things calm down or a face-to-face meeting
is essential.}

\section{Description, Content, and Learning Goals}

Introduction to Programming teaches basic programming skills that are applicable to a variety of disciplines and also acts as a bridge to continued studies in Computer Science. Students will work with the Python programming language in order to solve basic problems involving digital media: images, sound.  By the end of the course students will be able to read and develop computer programs utilizing the following programming concepts: basic data types and encoding, variables and scope, array and list data structures, if statements and conditional execution, loops and iteration, functions, and object types.

\subsection{Textbook}

\noindent
Guadial, Mark J. \& Ericson, Barbara. \textit{Introduction to Computing and Programming in Python: A Multimedia Approach. Fourth Edition}. Pearson. Hoboken, NJ. 2016. % chktex 8

\subsection{Software}

All programming will be done using \textit{Jython Environment for Students (JES)}. This software is avaiable from any campus computer. It is also free to download and install. A link to the latest version will be posted to Moodle and the instructor is happy to help with getting it installed and setup as needed.

\section{Workload}
% number of/details on midterms, finals, project, homeworks, quizes, etc

The course workload is as follows:
\begin{center}
  \begin{tabular}{ll}
    \underline{Category} & \underline{Number of Assignments} \\
    Exams & 6 \\
    Projects & 2 \\
    Labs & 10 \\
    Homework & 8
  \end{tabular}
\end{center}


\subsection*{Exams}

All exams are weighted equally. There is no midterm or final exam in the sense that the exams are worth more than other exams or that they will necessarily take longer than other exams.  Exams will generally focus on material covered since the previous exam but will be in some sense cumulative due to the nature of programming. Exams will be not be monitored nor run during class periods. They will be assigned with a day or two at most to complete and you will be expected to work on your own and adhere to any restrictions (no book, etc.) listed on the exam. Failing to adhere to posted exam restrictions will be treated as a case of academic dishonesty and may result in a zero for the exam or the course.

\subsection*{Projects}

Two larger scale programming projects will be undertaken during the semester. These projects will be individual efforts, but advice and consultation with the instructor and classmates is generally encouraged. They will require much more effort than the programs written in lab or as part of homework. Students can expect to have two weeks from the time of the project assignment to complete the project. One or more lab periods will be dedicated to work on the project. It is highly recommend that all students make ample use of the time given on these projects.

\subsection*{Homework}

Students will be assigned a set of problems from each chapter of the book covered in the course. These problems are meant to guide reading, prepare the student for in class problems, and survey the material covered by the exam.

\subsection*{Labs}

Lab assignments are meant to be completed during the two hour lab period. For a variety of reasons, this isn't always possible. When completion is not possible, the goal is to make good constructive progress on the assignment. Full credit can and will be given on unfinished work so long as it can be executed to complete some portion of the given task, shows evidence of purposeful progress, and the group made full use of the lab period. Submitted coding work should always run without crashing but may not complete the task as specified. It could be paritially done or carry out a stepping stone task along the path to given problem. One of your major goals in lab should be to develop habits for working in small, functional steps with your code, and to develop strategies for fixing or working around problematic bugs.

\subsection{Course Engagement Expectations}

The weekly workload for this course will vary by student but on average should be about 13 hours per week.  The follow tables provides a rough estimate of the distribution of this time over different course components.
\begin{center}
\begin{tabular}{ll}
\underline{Assignment Type} & \underline{Time/week} \\
Lectures+Labs       & 6 hours/week \\
Homework          & 1 hours/week \\
Exam Study Time    & 0.5 hours/week \\
Projects          & 3 hours/week \\
Reading &  2.5 hours/week \\
\bottomrule
 & 13 hours/week
\end{tabular}
\end{center}

\section{Grades}

This course uses a standard grading scale where percentage grades translate to letter grades as follows:

\begin{center}
\begin{small}
\begin{tabular}{lcl}
\underline{Score} & & \underline{Grade} \\
94--100 & & A \\
90--93 & & A- \\
88--89 & & B+ \\
82--87 & & B \\
80--81 & & B- \\
78--79 & & C+ \\
72--77 & & C \\
70--71 & & C- \\
68--69 & & D+ \\
62--67 & & D \\
60--61 & & D- \\
0--59 & & F
\end{tabular}
\end{small}
\end{center}


Students are always welcome to challenge a grade that they feel is unfair or calculated incorrectly.  Mistakes made in the student's favor will never be corrected to lower a grade.  Mistakes not in the student's favor will be corrected.  \textit{Basically, after the initial grading, a score can only go up as the result of a challenge.}



\subsection{Grade Weights}

The final grade is based on a weighted average of particular assignment categories.

\begin{center}
  \begin{tabular}{ll}
  \underline{Category} & \underline{Weight} \\
    Exams & 36\% \\ %6 each
    Projects & 24\% \\ %12 each
    Homework & 10\% \\ %1.25 each
    Labs & 15\% \\ %1.5 each
    Participation & 15\%
  \end{tabular}
\end{center}

\subsection{Participation \& Attendance}

Class and lab attendance will be monitored via Zoom. We will also make good use of polling and interactive feedback software and regular participation in these activities will also factor into your participation grade. Repeated unexcused absenses will have a negative effect on your participation grade. Whenever possible, let the instructor know of the absence before it occurs. When unexcused absences do occur, it is the student's responsibility to make up for the lost class time and to seek the permission of the instructor to hand-in or complete assignments that are late due to an unexcused absence.

\subsection{Late Work}

In general, assignments are due at the specified time and no late assignments will be accepted unless an extension was requested prior to the due date. There are, of course, exceptions to this rule and students needing extra time can always contact the instructor for an extension. Do not just give up and eat a zero for the assignment. Ever. There is no penalty in asking for an extension nor is there a limit on extensions.  That being said, there is no guarantee an extension will be given without legitimate need.

\subsection{Academic Honesty}

From the Monmouth College Academic Honesty Policy:
\begin{quote}
  ``We view academic dishonesty as a threat to the integrity and intellectual mission of our institution. Any breach of the academic honesty policy - either intentionally or unintentionally - will be taken seriously and may result not only in failure in the course, but in suspension or expulsion from the college. It is each student’s responsibility to read, understand and comply with the general academic honesty policy at Monmouth College, as defined here in the Scots Guide, and to the specific guidelines for each course, as elaborated on the professor’s syllabus.''

  ``The following areas are examples of violations of the academic honesty policy:
  \begin{enumerate}
  \item Cheating on tests, labs, etc;
  \item Plagiarism, i.e., using the words, ideas, writing, or work of another without giving appropriate credit;
  \item Improper collaboration between students, i.e., not doing one’s own work on outside assignments specified as group projects by the instructor;
  \item Submitting work previously submitted in another course, without previous authorization by the instructor.''
  \end{enumerate}

  ``Please note that this list is not intended to be exhaustive.''
\end{quote}

The complete Monmouth College Academic Honesty Policy can be found on the College web page by clicking on ``Student Life'' then on ``Scot’s Guide'' in the navigation bar to the left, then ``Academic Regulations'' in the navigation bar at the left.  Or you can visit the web page directly by typing in this URL: \url{https://ou.monmouthcollege.edu/life/residence-life/scots-guide/academic-regulations.aspx}

In this course, any violation of the academic honesty policy will have varying consequences depending on the severity of the infraction as judged by the instructor. Minimally, a violation will result in an``F'' or 0 points on the assignment in question. Additionally, the student’s course grade may be lowered by one letter grade. In severe cases, the student will be assigned a course grade of ``F'' and dismissed from the class. All cases of academic dishonesty will be reported to the Associate Dean who may decide to recommend further action to the Admissions and Academic Status Committee, including suspension or dismissal. It is assumed that students will educate themselves regarding what is considered to be academic dishonesty, so excuses or claims of ignorance will not mitigate the consequences of any violations.

\section{Accessibility}

Student Success \& Accessibility Services offers FREE resources to assist Monmouth College students with their academic success. Programs include Supplemental Instruction for select classes, Drop-In and appointment tutoring, and individual Academic Coaching. Our office is here to help all students excel academically, since all students can work toward better grades, practice stronger study skills, and manage their time better.

If you have a disability or had academic accommodations in high school or another college, you may be eligible for academic accommodations at Monmouth College under the Americans with Disabilities Act (ADA). Monmouth College is committed to equal educational access. To discuss any of the services offered, please call or meet with Robert Crawley, Interim Director of Student Success \& Accessibility Services.  SSAS is located in the new ACE space on the first floor of the Hewes Library, opposite Einstein’s Bros Bagels. They can be reached at 309-457-2257 or via email at: ssas@monmouthcollege.edu.

\subsection{Calendar}

\textit{This calendar is subject to change based on the circumstances of the course.}

\begin{center}
\begin{tabular}{llll}
\underline{Week} & \underline{Dates} & \underline{Assignments Due} & \underline{Chapter(s)}\\
1 & 1/25 --- 1/29  &  &  1\\
2 & 2/1 --- 2/5 & Lab 1. Hwk 1. & 2,4\\
3 & 2/8 --- 2/12 & Lab 2. Hwk 2. Exam 1.  &  4  \\
4 & 2/15 --- 2/19  & Lab 3. Hwk 3. &   4,5 \\
5 & 2/22 --- 2/26 & Lab 4. Exam 2. & 5\\
6 & 3/1 --- 3/5 & Lab 5. Hwk 4. . & 5,6\\
7 & 3/8 --- 3/12 & Lab 6. Exam 3 . &   6 \\
8 & 3/15 --- 3/19 & Lab 7.  Hwk 5. &  6 \\
9 & 3/22 --- 3/26  & Exam 4 & 10  \\
10 & 3/29 --- 4/2  & EASTER (F). Project 1. &  10, 7  \\
11 & 4/5 --- 4/9 &  Lab 8. Hwk 6. &  7,8 \\
12 & 4/12 --- 4/16 &  Lab 9. Hwk 7. & 8,9 \\
13 & 4/19 --- 4/23 & Lab 10. Exam 5.  &  9 \\
14 & 4/26 --- 4/30 &  Project 2. &  15 \\
15 & 5/3 --- 5/8 & FINALS BEGIN (F,Sat)  &  \\
16 & 5/10 -- 5/12 & (EXAMS) Exam 6. &  \\
\end{tabular}
\end{center}

\end{document}
